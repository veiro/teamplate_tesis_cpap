\setglossarypreamble[glosario]{Un glosario incluye una lista de términos y su explicación sucinta. El objetivo de este apartado es permitirle a un lector especializado en el área, aunque no necesariamente en la temática, comprender con mayor facilidad ciertos términos. 

Se organiza en forma alfabética y en el cuerpo de la obra se lo puede señalar con \textsc{versalita} la primera vez que se menciona, si este tipo de letra no fue utilizado con otro fin.
 }



\longnewglossaryentry{adjetivo}
{
type={glosario},
name={Adjetivo},
text={\textsc{Adjetivo}}
}
{Clase de palabra definida mediante los rasgos [+N +V] y caracterizada semánticamente por expresar una cualidad.
}

\longnewglossaryentry{adjetivo_cu}
{
type={glosario},
name={Adjetivo~cualitativo},
text={\textsc{Adjetivo cualitativo}},
}
{Se denomina así a adjetivos como \textit{alto}, \textit{bueno}, etc., que predican una propiedad del nombre al que van asociados como modificadores.
}

\longnewglossaryentry{adjetivo_re}
{
type={glosario},
name={Adjetivo~relacional},
text={\textsc{Adjetivo relacional}}
}
{Se denomina así a adjetivos como \textit{español}, \textit{constitucional}, etc., que se caracterizan por ir siempre pospuestos al sustantivo y por poseer propiedades referenciales: donotan un objeto del mundo y funcionan como un argumento del nombre.
}




